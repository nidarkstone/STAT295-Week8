% Options for packages loaded elsewhere
\PassOptionsToPackage{unicode}{hyperref}
\PassOptionsToPackage{hyphens}{url}
%
\documentclass[
]{article}
\usepackage{amsmath,amssymb}
\usepackage{iftex}
\ifPDFTeX
  \usepackage[T1]{fontenc}
  \usepackage[utf8]{inputenc}
  \usepackage{textcomp} % provide euro and other symbols
\else % if luatex or xetex
  \usepackage{unicode-math} % this also loads fontspec
  \defaultfontfeatures{Scale=MatchLowercase}
  \defaultfontfeatures[\rmfamily]{Ligatures=TeX,Scale=1}
\fi
\usepackage{lmodern}
\ifPDFTeX\else
  % xetex/luatex font selection
\fi
% Use upquote if available, for straight quotes in verbatim environments
\IfFileExists{upquote.sty}{\usepackage{upquote}}{}
\IfFileExists{microtype.sty}{% use microtype if available
  \usepackage[]{microtype}
  \UseMicrotypeSet[protrusion]{basicmath} % disable protrusion for tt fonts
}{}
\makeatletter
\@ifundefined{KOMAClassName}{% if non-KOMA class
  \IfFileExists{parskip.sty}{%
    \usepackage{parskip}
  }{% else
    \setlength{\parindent}{0pt}
    \setlength{\parskip}{6pt plus 2pt minus 1pt}}
}{% if KOMA class
  \KOMAoptions{parskip=half}}
\makeatother
\usepackage{xcolor}
\usepackage[margin=1in]{geometry}
\usepackage{color}
\usepackage{fancyvrb}
\newcommand{\VerbBar}{|}
\newcommand{\VERB}{\Verb[commandchars=\\\{\}]}
\DefineVerbatimEnvironment{Highlighting}{Verbatim}{commandchars=\\\{\}}
% Add ',fontsize=\small' for more characters per line
\usepackage{framed}
\definecolor{shadecolor}{RGB}{248,248,248}
\newenvironment{Shaded}{\begin{snugshade}}{\end{snugshade}}
\newcommand{\AlertTok}[1]{\textcolor[rgb]{0.94,0.16,0.16}{#1}}
\newcommand{\AnnotationTok}[1]{\textcolor[rgb]{0.56,0.35,0.01}{\textbf{\textit{#1}}}}
\newcommand{\AttributeTok}[1]{\textcolor[rgb]{0.13,0.29,0.53}{#1}}
\newcommand{\BaseNTok}[1]{\textcolor[rgb]{0.00,0.00,0.81}{#1}}
\newcommand{\BuiltInTok}[1]{#1}
\newcommand{\CharTok}[1]{\textcolor[rgb]{0.31,0.60,0.02}{#1}}
\newcommand{\CommentTok}[1]{\textcolor[rgb]{0.56,0.35,0.01}{\textit{#1}}}
\newcommand{\CommentVarTok}[1]{\textcolor[rgb]{0.56,0.35,0.01}{\textbf{\textit{#1}}}}
\newcommand{\ConstantTok}[1]{\textcolor[rgb]{0.56,0.35,0.01}{#1}}
\newcommand{\ControlFlowTok}[1]{\textcolor[rgb]{0.13,0.29,0.53}{\textbf{#1}}}
\newcommand{\DataTypeTok}[1]{\textcolor[rgb]{0.13,0.29,0.53}{#1}}
\newcommand{\DecValTok}[1]{\textcolor[rgb]{0.00,0.00,0.81}{#1}}
\newcommand{\DocumentationTok}[1]{\textcolor[rgb]{0.56,0.35,0.01}{\textbf{\textit{#1}}}}
\newcommand{\ErrorTok}[1]{\textcolor[rgb]{0.64,0.00,0.00}{\textbf{#1}}}
\newcommand{\ExtensionTok}[1]{#1}
\newcommand{\FloatTok}[1]{\textcolor[rgb]{0.00,0.00,0.81}{#1}}
\newcommand{\FunctionTok}[1]{\textcolor[rgb]{0.13,0.29,0.53}{\textbf{#1}}}
\newcommand{\ImportTok}[1]{#1}
\newcommand{\InformationTok}[1]{\textcolor[rgb]{0.56,0.35,0.01}{\textbf{\textit{#1}}}}
\newcommand{\KeywordTok}[1]{\textcolor[rgb]{0.13,0.29,0.53}{\textbf{#1}}}
\newcommand{\NormalTok}[1]{#1}
\newcommand{\OperatorTok}[1]{\textcolor[rgb]{0.81,0.36,0.00}{\textbf{#1}}}
\newcommand{\OtherTok}[1]{\textcolor[rgb]{0.56,0.35,0.01}{#1}}
\newcommand{\PreprocessorTok}[1]{\textcolor[rgb]{0.56,0.35,0.01}{\textit{#1}}}
\newcommand{\RegionMarkerTok}[1]{#1}
\newcommand{\SpecialCharTok}[1]{\textcolor[rgb]{0.81,0.36,0.00}{\textbf{#1}}}
\newcommand{\SpecialStringTok}[1]{\textcolor[rgb]{0.31,0.60,0.02}{#1}}
\newcommand{\StringTok}[1]{\textcolor[rgb]{0.31,0.60,0.02}{#1}}
\newcommand{\VariableTok}[1]{\textcolor[rgb]{0.00,0.00,0.00}{#1}}
\newcommand{\VerbatimStringTok}[1]{\textcolor[rgb]{0.31,0.60,0.02}{#1}}
\newcommand{\WarningTok}[1]{\textcolor[rgb]{0.56,0.35,0.01}{\textbf{\textit{#1}}}}
\usepackage{longtable,booktabs,array}
\usepackage{calc} % for calculating minipage widths
% Correct order of tables after \paragraph or \subparagraph
\usepackage{etoolbox}
\makeatletter
\patchcmd\longtable{\par}{\if@noskipsec\mbox{}\fi\par}{}{}
\makeatother
% Allow footnotes in longtable head/foot
\IfFileExists{footnotehyper.sty}{\usepackage{footnotehyper}}{\usepackage{footnote}}
\makesavenoteenv{longtable}
\usepackage{graphicx}
\makeatletter
\def\maxwidth{\ifdim\Gin@nat@width>\linewidth\linewidth\else\Gin@nat@width\fi}
\def\maxheight{\ifdim\Gin@nat@height>\textheight\textheight\else\Gin@nat@height\fi}
\makeatother
% Scale images if necessary, so that they will not overflow the page
% margins by default, and it is still possible to overwrite the defaults
% using explicit options in \includegraphics[width, height, ...]{}
\setkeys{Gin}{width=\maxwidth,height=\maxheight,keepaspectratio}
% Set default figure placement to htbp
\makeatletter
\def\fps@figure{htbp}
\makeatother
\usepackage{soul}
\setlength{\emergencystretch}{3em} % prevent overfull lines
\providecommand{\tightlist}{%
  \setlength{\itemsep}{0pt}\setlength{\parskip}{0pt}}
\setcounter{secnumdepth}{5}
\ifLuaTeX
  \usepackage{selnolig}  % disable illegal ligatures
\fi
\IfFileExists{bookmark.sty}{\usepackage{bookmark}}{\usepackage{hyperref}}
\IfFileExists{xurl.sty}{\usepackage{xurl}}{} % add URL line breaks if available
\urlstyle{same}
\hypersetup{
  pdftitle={STAT295-Week8},
  pdfauthor={Nida Karataş},
  hidelinks,
  pdfcreator={LaTeX via pandoc}}

\title{STAT295-Week8}
\author{Nida Karataş}
\date{2024-04-15}

\begin{document}
\maketitle

{
\setcounter{tocdepth}{2}
\tableofcontents
}
\hypertarget{introduction-to-r-markdown}{%
\section{Introduction to R markdown}\label{introduction-to-r-markdown}}

\hypertarget{r-markdown}{%
\subsection{R Markdown}\label{r-markdown}}

This is an \texttt{R}` \texttt{Markdown}` document. Markdown is a simple formatting syntax for authoring \emph{HTML}, \textbf{PDF}, and \textbf{MS} Word documents. For more details on using R Markdown see \url{http://rmarkdown.rstudio.com}.

When you click the \textbf{Knit} button a document will be generated that includes both content as well as the output of any embedded R code chunks within the document. You can embed an R code chunk like this:

This project includes some summary statistics:

\begin{Shaded}
\begin{Highlighting}[]
\FunctionTok{summary}\NormalTok{(cars)}
\end{Highlighting}
\end{Shaded}

\begin{verbatim}
##      speed           dist       
##  Min.   : 4.0   Min.   :  2.00  
##  1st Qu.:12.0   1st Qu.: 26.00  
##  Median :15.0   Median : 36.00  
##  Mean   :15.4   Mean   : 42.98  
##  3rd Qu.:19.0   3rd Qu.: 56.00  
##  Max.   :25.0   Max.   :120.00
\end{verbatim}

\hypertarget{include-plots}{%
\subsection{Include Plots}\label{include-plots}}

You can embed plots, for example:

\includegraphics{week8_files/figure-latex/pressure-1.pdf}

\hypertarget{list}{%
\subsection{List}\label{list}}

\begin{itemize}
\item
  bullet list 1
\item
  bullet list 2
\item
  bullet list 3
\end{itemize}

\hypertarget{format}{%
\subsection{Format}\label{format}}

\begin{itemize}
\item
  superscript: 2\textsuperscript{nd}
\item
  subscript: CO\textsubscript{2}
\item
  strike-through: \st{mistake}
\end{itemize}

\hypertarget{code-chunk-options}{%
\subsection{Code Chunk Options}\label{code-chunk-options}}

\begin{Shaded}
\begin{Highlighting}[]
\FunctionTok{sqrt}\NormalTok{(}\SpecialCharTok{{-}}\DecValTok{1}\NormalTok{)}
\end{Highlighting}
\end{Shaded}

\begin{verbatim}
## [1] NaN
\end{verbatim}

\hypertarget{inline-codes}{%
\subsection{Inline Codes}\label{inline-codes}}

I want to add some inline codes such as 5. There are 84 lines in CO2 data.The number of variables in CO2 data is 5. Here are the variables names are Plant, Type, Treatment, conc, uptake.

There are 150 lines in iris data.The number of variables in iris data is 5. Here are the variables names are Sepal.Length, Sepal.Width, Petal.Length, Petal.Width, Species.

\hypertarget{tables}{%
\subsection{Tables}\label{tables}}

\begin{longtable}[]{@{}lll@{}}
\caption{MyTable}\tabularnewline
\toprule\noalign{}
Col1 & Col2 & Col3 \\
\midrule\noalign{}
\endfirsthead
\toprule\noalign{}
Col1 & Col2 & Col3 \\
\midrule\noalign{}
\endhead
\bottomrule\noalign{}
\endlastfoot
& & \\
& & \\
& & \\
\end{longtable}

\begin{Shaded}
\begin{Highlighting}[]
\NormalTok{top\_gap }\OtherTok{\textless{}{-}} \FunctionTok{head}\NormalTok{(gapminder)}
\NormalTok{knitr}\SpecialCharTok{::}\FunctionTok{kable}\NormalTok{(top\_gap)}
\end{Highlighting}
\end{Shaded}

\begin{tabular}{l|l|r|r|r|r}
\hline
country & continent & year & lifeExp & pop & gdpPercap\\
\hline
Afghanistan & Asia & 1952 & 28.801 & 8425333 & 779.4453\\
\hline
Afghanistan & Asia & 1957 & 30.332 & 9240934 & 820.8530\\
\hline
Afghanistan & Asia & 1962 & 31.997 & 10267083 & 853.1007\\
\hline
Afghanistan & Asia & 1967 & 34.020 & 11537966 & 836.1971\\
\hline
Afghanistan & Asia & 1972 & 36.088 & 13079460 & 739.9811\\
\hline
Afghanistan & Asia & 1977 & 38.438 & 14880372 & 786.1134\\
\hline
\end{tabular}

\begin{Shaded}
\begin{Highlighting}[]
\NormalTok{knitr}\SpecialCharTok{::}\FunctionTok{kable}\NormalTok{(top\_gap,}
             \AttributeTok{caption =} \StringTok{"The first rows of the dataset, gapminder"}\NormalTok{)}
\end{Highlighting}
\end{Shaded}

\begin{table}

\caption{\label{tab:unnamed-chunk-4}The first rows of the dataset, gapminder}
\centering
\begin{tabular}[t]{l|l|r|r|r|r}
\hline
country & continent & year & lifeExp & pop & gdpPercap\\
\hline
Afghanistan & Asia & 1952 & 28.801 & 8425333 & 779.4453\\
\hline
Afghanistan & Asia & 1957 & 30.332 & 9240934 & 820.8530\\
\hline
Afghanistan & Asia & 1962 & 31.997 & 10267083 & 853.1007\\
\hline
Afghanistan & Asia & 1967 & 34.020 & 11537966 & 836.1971\\
\hline
Afghanistan & Asia & 1972 & 36.088 & 13079460 & 739.9811\\
\hline
Afghanistan & Asia & 1977 & 38.438 & 14880372 & 786.1134\\
\hline
\end{tabular}
\end{table}

\begin{Shaded}
\begin{Highlighting}[]
\NormalTok{knitr}\SpecialCharTok{::}\FunctionTok{kable}\NormalTok{(top\_gap,}
             \AttributeTok{caption =} \StringTok{"The first rows of the dataset, gapminder"}\NormalTok{)}
\end{Highlighting}
\end{Shaded}

\begin{table}

\caption{\label{tab:mytable1}The first rows of the dataset, gapminder}
\centering
\begin{tabular}[t]{l|l|r|r|r|r}
\hline
country & continent & year & lifeExp & pop & gdpPercap\\
\hline
Afghanistan & Asia & 1952 & 28.801 & 8425333 & 779.4453\\
\hline
Afghanistan & Asia & 1957 & 30.332 & 9240934 & 820.8530\\
\hline
Afghanistan & Asia & 1962 & 31.997 & 10267083 & 853.1007\\
\hline
Afghanistan & Asia & 1967 & 34.020 & 11537966 & 836.1971\\
\hline
Afghanistan & Asia & 1972 & 36.088 & 13079460 & 739.9811\\
\hline
Afghanistan & Asia & 1977 & 38.438 & 14880372 & 786.1134\\
\hline
\end{tabular}
\end{table}

To cite a table we can use Table \ref{tab:mytable1}

\end{document}
